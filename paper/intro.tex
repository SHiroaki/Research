\subsection{研究の目的}
現在の日常的に用いられている不規則動詞はおおよそ180語存在し、
会話の中に出現する動詞の約70\%が
不規則動詞である\cite{pinker}.これらの不規則動詞はOld English時代[AD 800 頃]に強変化動詞(strong verb)と呼ばれ主に母音が変化して活用していた.
Modern Englishでは例外も含め9クラスに分類され、クラス内にも細かい分類がある\cite{pinker2}.
しかし、不規則動詞には接尾辞[-ed]をつける規則的な活用に変化しているという現象が起こっている.
英語の歴史的な流れの中では、Old Englishから中期英語時代における海賊によるイングランドの侵略、ノルマン征服などの
人口流入を伴った言語接触により不規則動詞の規則化の誘発、またその加速が起こっている.\\
 本研究ではこの不規則動詞の規則化に対する人口流入の規模、頻度をシミュレーションによって検証することを目的とする.検証のために遺伝的アルゴリズム\cite{iba_ga}(以下GA)をベースに、エージェントコミュニケーションと変化
を進行させるような(外圧)を組み込んだモデルを作成し、複数世代を通したシミュレーションを行う.

\subsection{本論文の構成}
