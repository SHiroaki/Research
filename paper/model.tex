本章ではマルチエージェント環境の中で各エージェントが
コミュニケーションを通して共通の過去形を獲得するモデル
を提案する.このモデルを使用することで、規則化の速度や
その速度を引き起こす原因について考察できると考えられる.


\subsection{実験に必要な知識}
本節では提案モデルに必要となる知識について述べる.
マルチエージェントとは何か、またコミュニケーションの
目的と言語変化への関わりについて述べる.
マルチエージェント環境を現実世界の人間のつながりに近づけるために
使用する複雑ネットワークについも説明を行う.
また、人工的な動詞の過去形をどうやって生成しているのかアンケート調査
を行った研究とそれに付随して単語の音韻的距離を計測するアルゴリズムに
ついて述べる.
最後にエージェントの学習機構に用いる遺伝的アルゴリズムについて述べる.

\subsubsection{マルチエージェントとコミュニケーション}
\subsubsection{複雑ネットワークを用いた言語変化シミュレーション}
\subsubsection{人工的な不規則動詞に対する既存の母音活用の適用}
\subsubsection{MetaPhoneアルゴリズム}
\subsubsection{遺伝的アルゴリズム}

